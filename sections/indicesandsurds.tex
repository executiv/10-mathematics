\section{Indices and Surds}
\begin{outline}

\0
\subsection{Irrational numbers and surds}
	\1 Square factors
		\2 When a factor of a number is a perfect square it can be called a square factor.
			\3 Examples of square numbers
				\[4,\ 9,\ 16,\ 25,\ 36,\ 49,\ 64,\ 81,\ 100\dots\]
	\1 Real numbers
		\2 Real numbers are numbers that can be located on a number line as a point. They include rational numbers, which are numbers able to be represented as fractions. The decimal form of a rational number either ends in recurring digits or ends by way of termination. Rational numbers take the form of either fractions or decimals. Irrational numbers, are numbers that cannot be expressed as fractions. The decimal representation of an irrational number is an infinite non-recurring decimal. Irrational numbers cannot be expressed in the form $\frac{a}{b}$ where $a$ and $b$ are integers and $b \ne 0$.
			\3 Rational numbers
				\[\frac{3}{7}, {-\frac{4}{39}}, -3, 1.6, 2.\dot7, 0.\overline{19}\]
			\3 Irrational numbers
				\[\sqrt{3}, {-2}\sqrt{7}, \sqrt{12} - 1, \pi, 2\pi - 3\]
	\1 Surds
		\2 Surds are irrational numbers that use a root sign.
			\3 These are surds:
				\[\sqrt{2}, \sqrt[5]11, {-\sqrt{200}}, 1 + \sqrt{5}\]
			\3 These are not surds:
				\[\sqrt{4}\ (=2), \sqrt[3]{125}\ (=5), -\sqrt[4]{16}\ (=-4)\]
	\1 Rules that apply to surds
		\[(\sqrt{x})^2 = x \text{\ and\ } \sqrt{x^2} = x \text{\ \ \ \ if\ \ \ \ } x \geq 0\]
		\[\sqrt{xy} = \sqrt{x} \times \sqrt{y} \text{\ \ \ \ if\ \ \ \ } x \geq 0 \text{\ and\ } y \geq 0\]
		\[\sqrt{\frac{x}{y}} = \frac{\sqrt{x}}{\sqrt{y}} \text{\ \ \ \ if\ \ \ \ } x \geq 0 \text{\ and\ } y > 0\]
	\1 Deciding if a number is rational or irrational
		\2 The method for finding whether a number is rational or irrational involves expressing the number as a decimal. If there is no recurring pattern, and the decimal doesn't terminate, one is dealing with an irrational number. If either of those conditions aren't followed, the number must be rational, and therefore not a surd.
	\1 Simplifying surds
		\2 To simplify a surd, one must identify what type of surd they are trying to simplify. There are four common types of surds, all which are simplified using different methods.
			\3 $\sqrt{x}$ surd - simplified by splitting the radicand into the highest square factor and the other factor the radicand divides into. This can then be split into two square roots - one which is a surd, and one which can be simplified. After simplifying the second square root, the expression can be shown as $y\sqrt{x}$ where $x$ is the surd and $y$ is the simplified square root. This is not to be confused with $\sqrt[y]{x}$, which denotes the $y$-th root of $x$ instead of $y$ multiplied by the square root of {x}.
				\[\sqrt{32} = \sqrt{16 \times 2}\]
				\[= \sqrt{16} \times \sqrt{2}\]
				\[= 4\sqrt{2}\]
			\3 $y\sqrt{x}$ surd - simplified by splitting the radicand into the highest square factor and the other factor the radicand divides into. This can then be split into two square roots - one which is a surd, and one which can be simplified. After simplifying the second square root, the two numbers not in root form can be multiplied together to make the final form a simplified $y\sqrt{x}$.
				\[3\sqrt{200} = 3\sqrt{100 \times 2}\]
				\[= 3 \times \sqrt{100} \times \sqrt{2}\]
				\[= 3 \times 10 \times \sqrt{2} = 30\sqrt{2}\]
			\3 $\frac{y\sqrt{x}}{z}$ surd - simplified by splitting the radicand into the highest square factor and the other factor the radicand divides into. This can then be split into two square roots - one which is a surd, and one which can be simplified. After simplifying the second square root, the two numbers not in root form (in the numerator) can be multiplied together. If there are any factors that can be cancelled in the final expression, they must be cancelled to make the expression the simplest form.
				\[\frac{5\sqrt{40}}{6} = \frac{5\sqrt{4\times 10}}{6}\]
				\[= \frac{5 \times \sqrt{4} \times \sqrt{10}}{6}\]
				\[= \frac{\cancelto{5}{10} \sqrt{10}}{\cancelto{3}{6}}\]
				\[= \frac{5\sqrt{10}}{3}\]
			\3 The last common form of surd is trivial to solve by removing the fraction's root and applying the root to both the numerator and denominator. From there, the normal method of solving fraction-based surds can be used.
				\[\sqrt{\frac{x}{y}} = \frac{\sqrt{x}}{\sqrt{y}}\]
	\1 Expressing surds as a square root of a positive integer (entire surds)
		\2 To express surds as a square root of a positive integer, one must express the $y$ of $y\sqrt{x}$ as the square root of the square of y. Then the two square roots can be combined.
			\[2\sqrt{5} = \sqrt{4} \times \sqrt{5} = \sqrt{20}\]

\0
\subsection{Adding and subtracting surds}
	\1 Like surds
	 	\2 Like surds are multiples of the same surd. Like surds can be added and subtracted. Before attempting to add or subtract surds, one must simplify all surds.
	 	 	\3 Examples
	 	 	 	\[\sqrt{3}, -5\sqrt{3}, \sqrt{12} (= 2\sqrt{3}), 2\sqrt{75} (= 10\sqrt{3})\]
	\1 Adding or subtracting surds using like terms.
	 	\2 Adding and subtracting surds is very simple. It involves making the $\sqrt{x}$ in the $y\sqrt{x}$ surd form (where $y$ is either implied to be 1, or is represented by a different number) equal in each term one wants to add or subtract, then combining the $y$ by either addition or subtraction, while keeping $\sqrt{x}$ the same.
	 	 	\3 Examples
	 	 	 	\[2\sqrt{3} + 4\sqrt{3}\]
	 	 	 	\[= 6\sqrt{3}\]
	 	 	 	\[\]
	 	 	 	\[9\sqrt{5} - 4\sqrt{5}\]
	 	 	 	\[= 5\sqrt{5}\]
	\1 Simplifying surds in order to add or subtract
	 	\2 Surds must be simplified to have the same radicand and index before they can be added or subtracted. To do this, one must identify the type of surd they are dealing with, and follow the correct steps to change the form of the surd. \textit{Refer to ``Irrational numbers and surds'' for more information on simplifying surds.}

\0
\subsection{Multiplying and dividing surds}
	\1 Rules for multiplying and dividing surds
	 	\2 When multiplying surds, the result will, in most common situations, abide by these rules:
	 	 	\[\sqrt{x} \times \sqrt{y} = \sqrt{xy}\]
	 	 	\[\text{Generally: } a\sqrt{x} \times b\sqrt{y} = ab\sqrt{xy}\]
	 	\2 When dividing surds, the result will, in most common situations, abide by these rules:
	 	 	\[\frac{\sqrt{x}}{\sqrt{y}} = \sqrt{\frac{x}{y}}\]
	 	 	\[\text{Generally: } \frac{a\sqrt{x}}{b\sqrt{y}} = \frac{a}{b}\sqrt{\frac{x}{y}}\]
	 	\2 When expanding brackets, the distributive law applies to surds as much as any other algebra:
	 	 	\[a(b+c) = ab+ac\]
	\1 Simplifying a product of two surds
	 	\2 To simplify the product of two surds, one must utilise the rules for multiplying surds.
	 	 	\3 Examples
	 	 	 	\[\sqrt{2} \times \sqrt{3}\]
	 	 	 	\[= \sqrt{2 \times 3}\]
	 	 	 	\[= \sqrt{6}\]
	 	 	 	\[\]
	 	 	 	\[2\sqrt{3} \times 3\sqrt{15}\]
	 	 	 	\[= 2 \times 3 \times \sqrt{3 \times 15}\]
	 	 	 	\[= 6\sqrt{45}\]
	 	 	 	\[= 6\sqrt{9 \times 5}\]
	 	 	 	\[= 18\sqrt{5}\]
	\1 Simplifying surds using division
	 	\2 To simplify surds using division, one must utilise the rules for dividing surds.
	 	 	\3 Examples
	 	 	 	\[-\sqrt{10} \div \sqrt{2}\]
	 	 	 	\[= -\sqrt{\frac{10}{2}}\]
	 	 	 	\[= -\sqrt{5}\]
	 	 	 	\[\]
	 	 	 	\[\frac{12\sqrt{18}}{3\sqrt{3}}\]
	 	 	 	\[= \frac{12}{3}\sqrt{\frac{18}{3}}\]
	 	 	 	\[= 4\sqrt{6}\]
	\1 Using the distributive law
	 	\2 To use the distributive law, one must multiply the external multiplier by each term within the brackets.
	 		\3 Examples
	 			\[\sqrt{3}(3\sqrt{5}-\sqrt{6})\]
	 			\[= 3\sqrt{15}-\sqrt{18}\]
	 			\[= 3\sqrt{15}-\sqrt{9 \times 2}\]
	 			\[= 3\sqrt{15}-3\sqrt{2}\]
	 			\[\]
	 			\[-3\sqrt{6}(2\sqrt{10}-4\sqrt{6})\]
	 			\[= -6\sqrt{60}+12\sqrt{36}\]
	 			\[= -6\sqrt{4 \times 15}+12 \times 6\]
	 			\[= -12\sqrt{15}+72\]

\0
\subsection{Binomial products}
	\1 Expanding binomial products
		\2 To expand binomial products, one must use the distributive law. This says that the first, outer, inner, and last pairs ``F.O.I.L.'' multiplied individually, then added together is equal to the binomial product.
			\3 F.O.I.L
				\[\text{First: }(\textbf{a} + b)(\textbf{c} + d)\]
				\[\text{Outer: }(\textbf{a} + b)(c + \textbf{d})\]
				\[\text{Inner: }(a + \textbf{b})(\textbf{c} + d)\]
				\[\text{Last: }(a + \textbf{b})(c + \textbf{d})\]
			\3 Equation
				\[(a + b)(c + d) = ac + ad + bc + bd\]

			\3 Examples
				\[(4 + \sqrt{3})(\sqrt{3} - 2)\]
				\[= 4\sqrt{3} - 8 + 3 - 2\sqrt{3}\]
				\[= 2\sqrt{3} - 5\]
				\[\]
				\[(2\sqrt{5} - 1)(3\sqrt{5} + 4)\]
				\[= 6 \times 5 + 8\sqrt{5} - 3\sqrt{5} - 4\]
				\[= 30 + 5\sqrt{5} - 4\]
				\[= 26 + 5\sqrt{5}\]
	\1 Expanding perfect squares
		\2 Expanding perfect squares is done by removing the square, and representing the equation as a binomial product.  If the equation has a negative before expanding, in the form $(x - y)^2$, From there, the problem may be solved through the normal process of expanding binomial products.
			\[(x + y)^2 = (x + y)(x + y)\]
	 	 	\[(x - y)^2 -(x + y)(x - y)\]
 	 	 	\[\]
 	 	 	\[(a + b)^2\]
 	 	 	\[a^2 + ab + ba + b^2\]
	 	 	\[a^2 + 2ab + b^2\]
	 	 	\[\]
	 	 	\[(a - b)^2\]
	 	 	\[a^2 - ab - ba + b^2\]
	 	 	\[a^2 - 2ab + b^2\]

\0
\subsection{Rationalising the denominator}
	\1 Method of rationalising a denominator
		\2 Rationalising a denominator is to change the denominator of a surd to a rational number. This is achieved by multiplying by a number equivalent to 1. The number equivalent to 1 will almost always be the radical and radicand within the denominator over the radical and radicand within the denominator. As $\frac{x}{x}$ = 1, this will always be equivalent to 1.
			\3 Method
				\[\frac{x}{\sqrt{y}} = \frac{x}{\sqrt{y}} \times \frac{\sqrt{y}}{\sqrt{y}} = \frac{x\sqrt{y}}{y}\]
			\3 Examples
				\[\frac{2}{\sqrt{3}} = \frac{2}{\sqrt{3}} \times \frac{\sqrt{3}}{\sqrt{3}} = \frac{2\sqrt{3}}{3}\]

\0
\subsection{Index laws reiterated}
	\1 Index laws
		\2 The first law
			\[a^m \times a^n = a^{m+n}\]
		\2 The second law
			\[a^m \div a^n = \frac{a^m}{a^n} = a^{m-n}\]
		\2 The third law
			\[(a^m) ^n = a^{m \times n}\]
		\2 The fourth law
			\[(a \times b)^m = a^m \times b^m\]
		\2 The fifth law
			\[\left(\frac{a}{b}\right)^m = \frac{a^m}{b^m}\]
		\2 Zero powers
			\[a^0 = 1\]
	\1 Using the first law
		\2 The first index law allows one to multiply expressions with indices, by retaining the base, and adding together the indices.
			\3 Examples
				\[x^5 \times x^4 = x^9\]
				\[3a^2b \times 4ab^3 = 12a^3b^4\]
	\1 Using the second law
		\2 The second index law states the method for dividing expressions with indices is to retain the base, and subtract the indices.
			\3 Examples
				\[m^7 \div m^5 = m^2\]
				\[4x^2y^5 \div (8xy^2) = \frac{4}{8}xy^3 = \frac{1}{2}xy^3\]
	\1 Using the third law
		\2 The third index law describes the method for expanding expressions with parenthetical indices, where the parentheses contain a term with an index, through retaining the base and multiplying the indices.
			\3 Examples
				\[(x^3)^6 = x^{18}\]
				\[(3x^4)^4 = 81x^{16}\]
	\1 Using the fourth law
		\2 The fourth index law describes the method for expanding parenthesis with an exterior index, by distributing the index number across the bases.
			\3 Examples
				\[(6 \times 9)^3 = 6^3 \times 9^3\]
				\[(4x \times 3y)^4 = 4x^4 \times 3y^4\]
	\1 Using the fifth law
		\2 The fifth index law enables simplification and rearrangement of fractions within parenthesis where there is an exterior index.
			\3 Examples
				\[\left(\frac{3x}{2y}\right)^3 = \frac{3x^3}{2y^3}\]
				\[\left(\frac{3xy}{2}\right)^5 = \frac{243x^5y^5}{32}\]

\0
\subsection{Negative indices}
	\1 Rules of negative indices
		\[a^{-m} = \frac{1}{a^m}\]
		\[\frac{1}{a^{-m}} = a^m\]
	\1 Writing expressions using positive indices
		\2 To write an expression with negative indices in a form that uses positive indices, one must use the rules of negative indices, combined with the index laws where necessary.
			\3 Examples
				\[b^{-4} = \frac{1}{b^4}\]
				\[3x^{-4}y^2 = \frac{3y^2}{x^4}\]

\0
\subsection{Scientific notation}
	\1 Rules of scientific notation
		\2 A number written using scientific notation is of the form $a \times 10^m$ where $1 \leq a < 10$ and $m$ is an integer.
		\2 Significant figures are the number of digits counted from left to right, starting at the first non-zero. Significant figures are a useful measure of decimal point accuracy using scientific notation.
			\3 Examples
				\[2.019 \times 10^7 \text{ has 4 significant figures.}\]
				\[3.44426 \times 10^3 \text{ has 6 significant figures.}\]
	\1 Converting from scientific notation to a basic numeral
		\2 To convert from scientific notation to basic numerals, one must move the decimal point. The direction and degree to which the decimal point must be moved is dependent on the index, $m$, in $a \times 10^m$. If the index is positive, the decimal point must be moved to the right. Conversely, if the index is negative, the decimal point must be moved to the left. The index itself represents the number of places the decimal point must be moved.
			\3 Examples
				\[5.016 \times 10^5 = 501600\]
				\[3.2 \times 10^{-7} = 0.00000032\]
	\1 Converting from basic numerals to scientific notation
		\2 To convert from basic numerals to scientific notation, one must move the decimal point (in the reverse direction than converting from scientific notation to basic numerals), then add the notation that specifies the exponent and multiplier. First, one must choose the number of places to move the decimal point. As scientific notation specifies that the number be less than 10 and equal or greater than 1, this number can be found by counting the number of places between the current position and the position after the first non-zero digit. This number, which will be positive if counting towards the left, or negative if counting toward the right, will become the index, $m$, in $a \times 10^m$.
			\3 Examples
				\4 3 significant figures
					\[5218300 = 5.22 \times 10^6\]
					\[0.0042031 = 4.20 \times 10^{-3}\]
				\4 6 significant figures
					\[225502849 = 2.25503 \times 10^8\]
					\[0.0008456832 = 8.45683 \times 10^{-4}\]

\0
\subsection{Rational indices}
	 \1 Rules of rational indices
		\[a^{\frac{1}{n}} = \sqrt[n]{a}\]
		\[a^{\frac{m}{n}} = (\sqrt[n]{a})^m\]
	 	\2 Writing in index form
		 	 \3 To write a surd in index form, one must identify the rules of rational indices. Recalling that $a = a^1$, one can change $\sqrt[n]{a^m}$, where $n$ is the index, $a$ is the radicand, and $m$ represents the index of $a$, into $a^{\frac{m}{n}}$.
		 	 	 \4 Examples
	 	 	 	 	 \[\sqrt{15} = 15^{\frac{1}{2}}\]
	 	 	 	 	 \[\sqrt{7x^5} = 7^{\frac{1}{2}}x^{\frac{5}{2}}\]
	 	 	 	 	 \[3\sqrt[4]{x^7} = 3x^{\frac{7}{4}}\]
	 	 	 	 	 \[10\sqrt{10} = 10^{\frac{3}{2}}\]
	 	\2 Writing in surd form
	 	 	\3 To write a number in index form in surd form, one must identify the rules of rational indices. Using the fact that $a = a^1$, one can change a number in index form $a^{\frac{m}{n}}$, where $a$ would become the radicand, $m$ would become the index of $a$, and $n$ would become the index, into the form $\sqrt[n]{a^m}$.
	 	 	 	\4 Examples
	 	 	 	 	\[3^{\frac{1}{5}} = \sqrt[5]{3}\]
	 	 	 	 	\[5^{\frac{2}{3}} = \left(\sqrt[3]{5}\right)^2 = \sqrt[3]{25}\]

\0
\subsection{Exponential equations}
	\1 Simple exponential equations
	 	\2 Simple exponential equations take the form $a^x = b$, where $a > 0$ and $a \neq 1$.
	\1 Solving exponential equations
		\2 To solve exponential equations, one must exploit the fact: if $a^x = a^y$ then $x = y$. This allows simple removal of the base to find the unknown index.
			\3 Examples
				\[2^x = 16\]
				\[2^x = 2^4\]
				\[\therefore x = 4\]
				\[\]
				\[3^x = \frac{1}{9}\]
				\[3^x = \frac{1}{3^2}\]
				\[3^x = 3^{-2}\]
				\[\therefore x = -2\]
				\[\]
				\[25^x = 125\]
				\[(5^2)^x = 5^3\]
				\[5^{2x} = 5^3\]
				\[\therefore 2x = 3\]
				\[\therefore x = \frac{3}{2}\]
	\1 Solving exponential equations with a variable on both sides
		\2 To solve an exponential equation with a variable on both side, one must find a way to make the two bases equal, which will in turn allow for the bases to be removed and the unknown index to be found.
			\3 Examples
				\[3^{2x-1} = 27^x\]
				\[3^{2x-1} = (3^3)^x\]
				\[3^{2x - 1} = 3^{3x}\]
				\[\therefore 2x - 1 = 3x\]
				\[\therefore x = -1\]

\0
\subsection{Exponential growth and compound interest}
	\1 Necessary knowledge
		\2 Exponential growth and decay can be modelled by the rule $A = ka^t$, where $A$ is the amount, $k$ is the initial amount and $t$ is the time. If $a > 0$, exponential growth occurs. If $0 < a < 1$, exponential decay occurs.
		\2 For a growth rate of $r\%$ p.a., the base `a' is calculated using $a = 1 + \frac{r}{100}$.
		\2 For a decay rate of $r\%$ p.a., the base `a' is calculated using $a = 1 - \frac{r}{100}$.
	\1 The exponential formula
		\[A = A_0\left(1 \pm \frac{r}{100}\right)^n\]
		\2 $A$ is the amount
		\2 $A_0$ is the initial amount
		\2 $r$ is the rate expressed as a percentage
		\2 $n$ is the time
		\2 $\pm$ depends on whether the calculation is growth (+) or decay (-)
	\1 Compound interest
		\2 Compound interest involves adding any interest earned to the balance at the end of each year or other period. The rule for the investment amount (\$$A$) is given by:
			\[A = A_0\left(1 + \frac{r}{100}\right)^t\]
			\3 $A_0$ is the initial amount
			\3 $r$ is the interest rate expressed as a percentage
			\3 $t$ is the time
	\1 Writing exponential rules
		\2 Compound interest (growth):
		
			\$100000 in savings at a rate of 14\% per annum
			
			\[A = A_0\left(1 + \frac{r}{100}\right)^n\]
			\[A = 100000\left(1 + \frac{14}{100}\right)^n\]
			\[\therefore A = 100000\left(1.14\right)^n\]
		\2 Population decreasing (decay):
		
			City population of 50000 decreasing by 12\% per year
			
			\[P = P_0\left(1 - \frac{r}{100}\right)^n\]
			\[P = 50000\left(1 - \frac{12}{100}\right)^n\]
			\[\therefore P = 50000\left(0.88\right)^n\]
	\1 Applying exponential rules
		\2 Finding the value after $n$ time
			\3 To find the value, $V$, after $n$ time, one must substitute a real value into the place of $n$. The calculation of this equation will yield $A$.
				\4 Examples
					\[\text{Starting Value: } \$145000 \text{; Growth rate: } 9\%\]
					\[\text{Find $V$ if $n = 1$}\]
					\[V = V_0\left(1.09\right)^n\]
					\[V = 145000\left(1.09\right)^1\]
					\[\therefore V = 158050\]
					\[\]
					\[\text{Find $V$ if $n = 3$}\]
					\[V = V_0\left(1.09\right)^n\]
					\[V = 145000\left(1.09\right)^3\]
					\[\therefore V = 187779.21\]

\end{outline}
